% Options for packages loaded elsewhere
\PassOptionsToPackage{unicode}{hyperref}
\PassOptionsToPackage{hyphens}{url}
%
\documentclass[
]{article}
\usepackage{amsmath,amssymb}
\usepackage{lmodern}
\usepackage{ifxetex,ifluatex}
\ifnum 0\ifxetex 1\fi\ifluatex 1\fi=0 % if pdftex
  \usepackage[T1]{fontenc}
  \usepackage[utf8]{inputenc}
  \usepackage{textcomp} % provide euro and other symbols
\else % if luatex or xetex
  \usepackage{unicode-math}
  \defaultfontfeatures{Scale=MatchLowercase}
  \defaultfontfeatures[\rmfamily]{Ligatures=TeX,Scale=1}
\fi
% Use upquote if available, for straight quotes in verbatim environments
\IfFileExists{upquote.sty}{\usepackage{upquote}}{}
\IfFileExists{microtype.sty}{% use microtype if available
  \usepackage[]{microtype}
  \UseMicrotypeSet[protrusion]{basicmath} % disable protrusion for tt fonts
}{}
\makeatletter
\@ifundefined{KOMAClassName}{% if non-KOMA class
  \IfFileExists{parskip.sty}{%
    \usepackage{parskip}
  }{% else
    \setlength{\parindent}{0pt}
    \setlength{\parskip}{6pt plus 2pt minus 1pt}}
}{% if KOMA class
  \KOMAoptions{parskip=half}}
\makeatother
\usepackage{xcolor}
\IfFileExists{xurl.sty}{\usepackage{xurl}}{} % add URL line breaks if available
\IfFileExists{bookmark.sty}{\usepackage{bookmark}}{\usepackage{hyperref}}
\hypersetup{
  pdftitle={Current State of Expungeable Records},
  hidelinks,
  pdfcreator={LaTeX via pandoc}}
\urlstyle{same} % disable monospaced font for URLs
\usepackage[margin=1in]{geometry}
\usepackage{color}
\usepackage{fancyvrb}
\newcommand{\VerbBar}{|}
\newcommand{\VERB}{\Verb[commandchars=\\\{\}]}
\DefineVerbatimEnvironment{Highlighting}{Verbatim}{commandchars=\\\{\}}
% Add ',fontsize=\small' for more characters per line
\usepackage{framed}
\definecolor{shadecolor}{RGB}{248,248,248}
\newenvironment{Shaded}{\begin{snugshade}}{\end{snugshade}}
\newcommand{\AlertTok}[1]{\textcolor[rgb]{0.94,0.16,0.16}{#1}}
\newcommand{\AnnotationTok}[1]{\textcolor[rgb]{0.56,0.35,0.01}{\textbf{\textit{#1}}}}
\newcommand{\AttributeTok}[1]{\textcolor[rgb]{0.77,0.63,0.00}{#1}}
\newcommand{\BaseNTok}[1]{\textcolor[rgb]{0.00,0.00,0.81}{#1}}
\newcommand{\BuiltInTok}[1]{#1}
\newcommand{\CharTok}[1]{\textcolor[rgb]{0.31,0.60,0.02}{#1}}
\newcommand{\CommentTok}[1]{\textcolor[rgb]{0.56,0.35,0.01}{\textit{#1}}}
\newcommand{\CommentVarTok}[1]{\textcolor[rgb]{0.56,0.35,0.01}{\textbf{\textit{#1}}}}
\newcommand{\ConstantTok}[1]{\textcolor[rgb]{0.00,0.00,0.00}{#1}}
\newcommand{\ControlFlowTok}[1]{\textcolor[rgb]{0.13,0.29,0.53}{\textbf{#1}}}
\newcommand{\DataTypeTok}[1]{\textcolor[rgb]{0.13,0.29,0.53}{#1}}
\newcommand{\DecValTok}[1]{\textcolor[rgb]{0.00,0.00,0.81}{#1}}
\newcommand{\DocumentationTok}[1]{\textcolor[rgb]{0.56,0.35,0.01}{\textbf{\textit{#1}}}}
\newcommand{\ErrorTok}[1]{\textcolor[rgb]{0.64,0.00,0.00}{\textbf{#1}}}
\newcommand{\ExtensionTok}[1]{#1}
\newcommand{\FloatTok}[1]{\textcolor[rgb]{0.00,0.00,0.81}{#1}}
\newcommand{\FunctionTok}[1]{\textcolor[rgb]{0.00,0.00,0.00}{#1}}
\newcommand{\ImportTok}[1]{#1}
\newcommand{\InformationTok}[1]{\textcolor[rgb]{0.56,0.35,0.01}{\textbf{\textit{#1}}}}
\newcommand{\KeywordTok}[1]{\textcolor[rgb]{0.13,0.29,0.53}{\textbf{#1}}}
\newcommand{\NormalTok}[1]{#1}
\newcommand{\OperatorTok}[1]{\textcolor[rgb]{0.81,0.36,0.00}{\textbf{#1}}}
\newcommand{\OtherTok}[1]{\textcolor[rgb]{0.56,0.35,0.01}{#1}}
\newcommand{\PreprocessorTok}[1]{\textcolor[rgb]{0.56,0.35,0.01}{\textit{#1}}}
\newcommand{\RegionMarkerTok}[1]{#1}
\newcommand{\SpecialCharTok}[1]{\textcolor[rgb]{0.00,0.00,0.00}{#1}}
\newcommand{\SpecialStringTok}[1]{\textcolor[rgb]{0.31,0.60,0.02}{#1}}
\newcommand{\StringTok}[1]{\textcolor[rgb]{0.31,0.60,0.02}{#1}}
\newcommand{\VariableTok}[1]{\textcolor[rgb]{0.00,0.00,0.00}{#1}}
\newcommand{\VerbatimStringTok}[1]{\textcolor[rgb]{0.31,0.60,0.02}{#1}}
\newcommand{\WarningTok}[1]{\textcolor[rgb]{0.56,0.35,0.01}{\textbf{\textit{#1}}}}
\usepackage{longtable,booktabs,array}
\usepackage{calc} % for calculating minipage widths
% Correct order of tables after \paragraph or \subparagraph
\usepackage{etoolbox}
\makeatletter
\patchcmd\longtable{\par}{\if@noskipsec\mbox{}\fi\par}{}{}
\makeatother
% Allow footnotes in longtable head/foot
\IfFileExists{footnotehyper.sty}{\usepackage{footnotehyper}}{\usepackage{footnote}}
\makesavenoteenv{longtable}
\usepackage{graphicx}
\makeatletter
\def\maxwidth{\ifdim\Gin@nat@width>\linewidth\linewidth\else\Gin@nat@width\fi}
\def\maxheight{\ifdim\Gin@nat@height>\textheight\textheight\else\Gin@nat@height\fi}
\makeatother
% Scale images if necessary, so that they will not overflow the page
% margins by default, and it is still possible to overwrite the defaults
% using explicit options in \includegraphics[width, height, ...]{}
\setkeys{Gin}{width=\maxwidth,height=\maxheight,keepaspectratio}
% Set default figure placement to htbp
\makeatletter
\def\fps@figure{htbp}
\makeatother
\setlength{\emergencystretch}{3em} % prevent overfull lines
\providecommand{\tightlist}{%
  \setlength{\itemsep}{0pt}\setlength{\parskip}{0pt}}
\setcounter{secnumdepth}{-\maxdimen} % remove section numbering
\ifluatex
  \usepackage{selnolig}  % disable illegal ligatures
\fi

\title{Current State of Expungeable Records}
\author{}
\date{\vspace{-2.5em}}

\begin{document}
\maketitle

\hypertarget{the-data}{%
\section{The data}\label{the-data}}

We should put some brief intro about the data we're using, though I
think they already know about it\ldots{}

\hypertarget{filter-to-expungeable-cases}{%
\section{Filter to expungeable
cases}\label{filter-to-expungeable-cases}}

\hypertarget{defining-the-cases-we-care-about}{%
\subsubsection{Defining the cases we care
about}\label{defining-the-cases-we-care-about}}

This is the most complicated section. Basically, I am defining a list of
the ``categories'' of cases that we care about, as defined by the new
law. Each ``category'' is stored as a query, which is a list with the
following elements:

\begin{itemize}
\tightlist
\item
  \texttt{reference} -- the section of the new law describing a specific
  category of case that is expungeable.
\item
  \texttt{title} -- a free text sample from that section of the new law,
  for explanatory purposes.
\item
  Any applicable filters. These elements each contain values that the
  specified field is allowed to take:

  \begin{itemize}
  \tightlist
  \item
    \texttt{CodeSection} -- The codes for the alleged crime
  \item
    \texttt{DispositionCode} -- The relevant dispositions of the cases
    for the relevant crimes
  \item
    \texttt{ChargeType} -- The relevant charge type (``Felony'',
    ``Misdemeanor'', etc.) if relevant
  \end{itemize}
\end{itemize}

\textbf{NOTE: this may be a partial list.} We should probably go through
the new law again with an LAJC lawyer to make sure we have coded this
correctly.

\begin{Shaded}
\begin{Highlighting}[]
\NormalTok{QUERIES }\OtherTok{\textless{}{-}} \FunctionTok{list}\NormalTok{(}
  \FunctionTok{list}\NormalTok{(}
    \AttributeTok{reference =} \StringTok{"19.2{-}392.6.A"}\NormalTok{,}
    \AttributeTok{title =} \StringTok{"Automatic sealing of offenses resulting in a deferred and dismissed disposition or conviction."}\NormalTok{,}
    \AttributeTok{CodeSection =} \FunctionTok{c}\NormalTok{(}\StringTok{"4.1{-}305"}\NormalTok{, }\StringTok{"18.2{-}250.1"}\NormalTok{),}
    \AttributeTok{DispositionCode =} \FunctionTok{c}\NormalTok{(}\StringTok{"Dismissed"}\NormalTok{) }
\NormalTok{  ),}
  \FunctionTok{list}\NormalTok{(}
    \AttributeTok{reference =} \StringTok{"19.2{-}392.6.B"}\NormalTok{,}
    \AttributeTok{title =} \StringTok{"Automatic sealing of offenses resulting in a deferred and dismissed disposition or conviction."}\NormalTok{,}
    \AttributeTok{CodeSection =} \FunctionTok{c}\NormalTok{(}\StringTok{"4.1{-}305"}\NormalTok{, }\StringTok{"18.2{-}96"}\NormalTok{, }\StringTok{"18.2{-}103"}\NormalTok{, }\StringTok{"18.2{-}119"}\NormalTok{, }\StringTok{"18.2{-}120"}\NormalTok{, }\StringTok{"18.2{-}134"}\NormalTok{),}
    \AttributeTok{DispositionCode =} \FunctionTok{c}\NormalTok{(}\StringTok{"Guilty"}\NormalTok{)}
\NormalTok{  ),}
  \FunctionTok{list}\NormalTok{(}
    \AttributeTok{reference =} \StringTok{"19.2{-}392.6.B{-}misdemeanor"}\NormalTok{,}
    \AttributeTok{title =} \StringTok{"Automatic sealing of offenses resulting in a deferred and dismissed disposition or conviction."}\NormalTok{,}
    \AttributeTok{CodeSection =} \FunctionTok{c}\NormalTok{(}\StringTok{"18.2{-}248.1"}\NormalTok{, }\StringTok{"18.2{-}250.1"}\NormalTok{, }\StringTok{"18.2{-}415"}\NormalTok{),}
    \AttributeTok{DispositionCode =} \FunctionTok{c}\NormalTok{(}\StringTok{"Guilty"}\NormalTok{),}
    \AttributeTok{ChargeType =} \FunctionTok{c}\NormalTok{(}\StringTok{"Misdemeanor"}\NormalTok{)}
\NormalTok{  ),}
  \FunctionTok{list}\NormalTok{(}
    \AttributeTok{reference =} \StringTok{"19.2{-}392.12"}\NormalTok{,}
    \AttributeTok{title =} \StringTok{"Sealing of offenses resulting in a deferred and dismissed disposition or conviction by petition."}\NormalTok{,  }
    \AttributeTok{CodeSection =} \FunctionTok{c}\NormalTok{(}\StringTok{"18.2{-}36.1"}\NormalTok{, }\StringTok{"18.2{-}36.2"}\NormalTok{, }\StringTok{"18.2{-}51.4"}\NormalTok{, }\StringTok{"18.2{-}51.5"}\NormalTok{, }\StringTok{"18.2{-}57.2"}\NormalTok{, }\StringTok{"18.2{-}266"}\NormalTok{, }\StringTok{"46.2{-}341.24"}\NormalTok{),}
    \AttributeTok{DispositionCode =} \FunctionTok{c}\NormalTok{(}\StringTok{"Deferred"}\NormalTok{, }\StringTok{"Dismissed"}\NormalTok{)}
\NormalTok{  )}
\NormalTok{)}
\end{Highlighting}
\end{Shaded}

Under the old law, any cases with the following dispositions should be
eligible for expungement.

\begin{Shaded}
\begin{Highlighting}[]
\NormalTok{OLD\_LAW }\OtherTok{\textless{}{-}} \FunctionTok{c}\NormalTok{(}
      \StringTok{"Dismissed"}\NormalTok{, }
      \StringTok{"Dismissed/Other"}\NormalTok{, }
      \StringTok{"Nolle Prosequi"}\NormalTok{, }
      \StringTok{"Not Guilty/Acquitted"}\NormalTok{, }
      \StringTok{"Not Guilty By Reason Of Insanity"}\NormalTok{, }
      \StringTok{"Not True Bill"}\NormalTok{, }
      \StringTok{"Not Guilty"}\NormalTok{, }
      \StringTok{"Not Guilty/Insanity"}
\NormalTok{    )}
\end{Highlighting}
\end{Shaded}

Here we map over the queries to get only the relevant cases that fit
\emph{any} of our queries. This is only the sample data. We pull in all
of the data below.

\begin{longtable}[]{@{}
  >{\raggedright\arraybackslash}p{(\columnwidth - 8\tabcolsep) * \real{0.26}}
  >{\raggedright\arraybackslash}p{(\columnwidth - 8\tabcolsep) * \real{0.13}}
  >{\raggedright\arraybackslash}p{(\columnwidth - 8\tabcolsep) * \real{0.17}}
  >{\raggedright\arraybackslash}p{(\columnwidth - 8\tabcolsep) * \real{0.13}}
  >{\raggedright\arraybackslash}p{(\columnwidth - 8\tabcolsep) * \real{0.32}}@{}}
\toprule
reference & CodeSection & DispositionCode & ChargeType & Charge \\
\midrule
\endhead
19.2-392.6.A & 4.1-305 & Dismissed & Misdemeanor & UNDERAGE POSS ALCO
21946 \\
19.2-392.6.A & 4.1-305 & Dismissed & Misdemeanor & UNDERAGE POSS OF
ALCOHOL \\
19.2-392.6.A & 18.2-250.1 & Dismissed & Misdemeanor & POSS OF MARIJ \\
19.2-392.6.A & 4.1-305 & Dismissed & Misdemeanor & POSS ALCOHOL BY MINOR
20949 \\
19.2-392.6.A & 4.1-305 & Dismissed & Misdemeanor & UNDERAGE POSS ALCO
ACC 20836 \\
19.2-392.6.B-misdemeanor & 18.2-415 & Guilty & Misdemeanor &
DISORDERLY \\
19.2-392.6.B-misdemeanor & 18.2-250.1 & Guilty & Misdemeanor & POSS
MARIJUANA 21682 \\
19.2-392.6.B-misdemeanor & 18.2-250.1 & Guilty & Misdemeanor & POSS
MARIJUANA 21519 \\
19.2-392.6.B-misdemeanor & 18.2-248.1 & Guilty & Misdemeanor & POSS
MARIJUANA W/INTENT 20252 \\
19.2-392.6.B-misdemeanor & 18.2-250.1 & Guilty & Misdemeanor & POSS
MARIJUANA 20067-01 \\
19.2-392.12 & 18.2-57.2 & Dismissed & Misdemeanor & ASSAULT: ON FAMILY
MEMBER \\
19.2-392.12 & 18.2-57.2 & Dismissed & Misdemeanor & ASSAULT \&
BATTERY \\
19.2-392.12 & 18.2-57.2 & Dismissed & Misdemeanor & ASSAULT AND
BATTERY \\
19.2-392.12 & 18.2-57.2 & Dismissed & Misdemeanor & ASSAULT AND
BATTER \\
19.2-392.12 & 18.2-57.2 & Dismissed & Misdemeanor & ASSAULT \\
19.2-392.12 & 18.2-57.2 & Dismissed & Misdemeanor & ASSAULT \\
19.2-392.12 & 18.2-57.2 & Dismissed & Misdemeanor & ASSAULT AND
BATTERY \\
19.2-392.12 & 18.2-57.2 & Dismissed & Misdemeanor & ASSAULT \&
BATTERY-FAM MEMBER \\
19.2-392.12 & 18.2-57.2 & Dismissed & Misdemeanor & ASSAULT \&
BATTERY-FAM MEMBER \\
19.2-392.12 & 18.2-57.2 & Dismissed & Misdemeanor & ASSAULT: ON FAMILY
MEMBER \\
19.2-392.12 & 18.2-57.2 & Dismissed & Misdemeanor & ASSAULT \&
BATTERY-FAM MEMBER \\
19.2-392.12 & 18.2-57.2 & Dismissed & Misdemeanor & ASSAULT \&
BATTERY-FAM MEMBER \\
19.2-392.12 & 18.2-57.2 & Dismissed & Misdemeanor & ASSAULT \&
BATTERY-FAM MEMBER \\
19.2-392.12 & 18.2-57.2 & Dismissed & Misdemeanor & ASSAULT \&
BATTERY \\
19.2-392.12 & 18.2-57.2 & Dismissed & Misdemeanor & ASSAULT \\
19.2-392.12 & 18.2-57.2 & Dismissed & Misdemeanor & ASSAULT AND
BATTER \\
19.2-392.12 & 18.2-57.2 & Dismissed & Misdemeanor & ASSAULT AND
BATTER \\
19.2-392.12 & 18.2-57.2 & Dismissed & Misdemeanor & ASSAULT AND
BATTER \\
19.2-392.12 & 18.2-57.2 & Dismissed & Misdemeanor & ASSAULT AND
BATTER \\
19.2-392.12 & 18.2-57.2 & Dismissed & Misdemeanor & ASSAULT \&
BATTERY-FAM MEMBER \\
19.2-392.12 & 18.2-57.2 & Dismissed & Misdemeanor & ASSAULT \&
BATTERY-FAM MEMBER \\
19.2-392.12 & 18.2-57.2 & Dismissed & Misdemeanor & ASSAULT AND
BATTER \\
19.2-392.12 & 18.2-57.2 & Dismissed & Misdemeanor & ASSAULT AND
BATTER \\
19.2-392.12 & 18.2-57.2 & Dismissed & Misdemeanor & ASSAULT \&
BATTERY-FAM MEMBER \\
19.2-392.12 & 18.2-57.2 & Dismissed & Misdemeanor & ASSAULT \&
BATTERY-FAM MEMBER \\
19.2-392.12 & 18.2-57.2 & Dismissed & Misdemeanor & ASSAULT: \\
19.2-392.12 & 18.2-57.2 & Dismissed & Misdemeanor & ASSAULT \&
BATTERY \\
19.2-392.12 & 18.2-57.2 & Dismissed & Misdemeanor & ASSAULT \&
BATTERY-FAM MEMBER \\
19.2-392.12 & 18.2-57.2 & Dismissed & Misdemeanor & ASSAULT \&
BATTERY \\
19.2-392.12 & 18.2-57.2 & Dismissed & Misdemeanor & ASSAULT: ON FAMILY
MEMBER \\
19.2-392.12 & 18.2-57.2 & Dismissed & Misdemeanor & ASSAULT \&
BATTERY-FAM MEMBER \\
19.2-392.12 & 18.2-57.2 & Dismissed & Misdemeanor & ASSAULT \&
BATTERY-FAM MEMBER \\
19.2-392.12 & 18.2-57.2 & Dismissed & Misdemeanor & ASSAULT \&
BATTERY-FAM MEMBER \\
19.2-392.12 & 18.2-57.2 & Dismissed & Misdemeanor & ASSAULT/BATTER \\
19.2-392.12 & 18.2-57.2 & Dismissed & Misdemeanor & ASSAULT/BATTER \\
19.2-392.12 & 18.2-57.2 & Dismissed & Misdemeanor & ASSAULT AND
BATTER \\
19.2-392.12 & 18.2-57.2 & Dismissed & Misdemeanor & JEFFRIES, HOWARD \\
19.2-392.12 & 18.2-57.2 & Dismissed & Misdemeanor & PUFFENBARGER,
MICHAEL JAMES \\
19.2-392.12 & 18.2-57.2 & Dismissed & Misdemeanor & ASSAULT \& BATTER \\
19.2-392.12 & 18.2-57.2 & Dismissed & Misdemeanor & ASSAULT: ON FAMILY
MEMBER \\
19.2-392.12 & 18.2-57.2 & Dismissed & Misdemeanor & ASSAULT: ON FAMILY
MEMBER \\
19.2-392.12 & 18.2-57.2 & Dismissed & Misdemeanor & ASSAULT \&
BATTERY-FAM MEMBER \\
19.2-392.12 & 18.2-57.2 & Dismissed & Misdemeanor & A/B PETER
DICKINSON \\
19.2-392.12 & 18.2-57.2 & Dismissed & Misdemeanor & ASSAULT \&
BATTERY \\
19.2-392.12 & 18.2-57.2 & Dismissed & Misdemeanor & ASSAULT: ON FAMILY
MEMBER \\
19.2-392.12 & 18.2-57.2 & Dismissed & Misdemeanor & ASSAULT \&
BATTERY-FAM MEMBER \\
19.2-392.12 & 18.2-57.2 & Dismissed & Misdemeanor & ASSAULT \\
19.2-392.12 & 18.2-57.2 & Dismissed & Misdemeanor & ASSAULT: ON FAMILY
MEMBER \\
19.2-392.12 & 18.2-57.2 & Dismissed & Misdemeanor & A\&B PATRICIA
GUILDS \\
\bottomrule
\end{longtable}

\hypertarget{the-full-data-summarized-by-year}{%
\subsection{The full data, summarized by
year}\label{the-full-data-summarized-by-year}}

This is where we roll it up to what they are likely looking for: by
year, how many cases are eligible for expungement and why? For the ``and
why?'' part, we give them the reference section in the new law.

\hypertarget{results-for-the-new-law}{%
\subsection{Results for the New Law}\label{results-for-the-new-law}}

Total counts of \emph{newly expungeable} cases by year:

\begin{longtable}[]{@{}lr@{}}
\toprule
year & n \\
\midrule
\endhead
2009 & 51348 \\
2010 & 50834 \\
2011 & 49932 \\
2012 & 50755 \\
2013 & 51464 \\
2014 & 49177 \\
2015 & 45464 \\
2016 & 41502 \\
2017 & 40723 \\
2018 & 42358 \\
2019 & 43002 \\
\bottomrule
\end{longtable}

\includegraphics{current-state-of-expungeable-records_files/figure-latex/unnamed-chunk-9-1.pdf}

\includegraphics{current-state-of-expungeable-records_files/figure-latex/unnamed-chunk-10-1.pdf}

\hypertarget{percentage-of-total-cases-that-are-expungeable-under-new-law-vs-old-law}{%
\subsection{Percentage of total cases that are expungeable (under new
law vs old
law)}\label{percentage-of-total-cases-that-are-expungeable-under-new-law-vs-old-law}}

This compares the number of additional cases that will be expungeable
under the new law vs.~the number of cases that were previously
expungeable (\emph{but should now be automatic?}).

\includegraphics{current-state-of-expungeable-records_files/figure-latex/unnamed-chunk-12-1.pdf}

The \texttt{Percent} column shows the percentage of total cases for that
year that fall into each category.

\begin{longtable}[]{@{}rrrr@{}}
\toprule
year & Eligibility & Count & Percent \\
\midrule
\endhead
2009 & new\_law & 51,348 & 2.0\% \\
2009 & old\_law & 542,491 & 20.6\% \\
2010 & new\_law & 50,834 & 2.0\% \\
2010 & old\_law & 528,055 & 20.6\% \\
2011 & new\_law & 49,932 & 2.0\% \\
2011 & old\_law & 512,694 & 20.7\% \\
2012 & new\_law & 50,755 & 2.1\% \\
2012 & old\_law & 493,657 & 20.4\% \\
2013 & new\_law & 51,464 & 2.2\% \\
2013 & old\_law & 485,856 & 20.3\% \\
2014 & new\_law & 49,177 & 2.1\% \\
2014 & old\_law & 466,768 & 20.1\% \\
2015 & new\_law & 45,464 & 2.1\% \\
2015 & old\_law & 448,924 & 20.9\% \\
2016 & new\_law & 41,502 & 2.0\% \\
2016 & old\_law & 432,458 & 21.3\% \\
2017 & new\_law & 40,723 & 2.0\% \\
2017 & old\_law & 435,669 & 20.9\% \\
2018 & new\_law & 42,358 & 2.1\% \\
2018 & old\_law & 432,647 & 21.0\% \\
2019 & new\_law & 43,002 & 2.0\% \\
2019 & old\_law & 459,581 & 21.2\% \\
\bottomrule
\end{longtable}

\end{document}
